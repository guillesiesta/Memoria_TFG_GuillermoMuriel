\chapter{Introducción}
El ser humano es considerado un ser curioso, inquieto y siempre en busca de la aventura. No hay más que pararse a observar a un niño con un año de edad y comprobarlo.

Nuestra especie necesita retos a los que enfrentarse para crecer y mejorar. En definitiva, abrir la mente para evolucionar.

El mundo en el que vivimos nos plantea diariamente una serie de obstáculos que tenemos que superar día a día. Este trabajo, se realiza la mayor parte de las veces de una manera mecánica. Es decir, durante esa actividad, el cerebro no está usándose a pleno rendimiento. No estamos motivados.

Sin embargo, con la aparición de los juegos, al ser humano se le ofrece una actividad distinta, se le permite crujir su rutina.

Volviendo al origen de nuestra especie, nos adentramos en nuestra principal característica evolutiva, la resolución de problemas, tan fundamental para el desarrollo de nuestro intelecto.

He aquí la idea principal del proyecto, que no es otra que la de generar esa posibilidad que ayude al ser humano a su desarrollo. Todo esto apoyado mediante la creación de un juego conversacional. Gracias al cual se planteen y resuelvan acertijos de los más ingeniosos: acertijos propuestos por individuos con alma de detective y resueltos por los mismos.

\section{Motivación y objetivo}

El poder colaborar en el desarrollo intelectual y en el entretenimiento de las personas es motivación más que suficiente para embarcarse en un proyecto de tal magnitud y dificultad.

Actualmente, cuando se decide crear un juego conversacional, se tienen al alcance de la mano, a través de internet, infinitas posibilidades para su creación.

El principal objetivo de este juego es la comunidad se pueda beneficiar del conocimiento incluido en el mismo. Para ello, la tarea fundamental será liberarlo.

Este juego será desarrollado de una manera actual, teniendo muy en cuenta filosofías de desarrollo de software ágiles y con eje fundamental el software libre. 

Con el objetivo de que la aplicación sea abierta, este proyecto estará liberado bajo la licencia \textit{GNU General Public License v3.0} \cite{licenciaproyecto} y alojado en un repositorio público en GitHub.

Este juego deberá ser accesible desde cualquier plataforma, por lo que para ello será fundamental su desarrollo mediante un servicio en la nube, es decir, el paradigma del ``\textit{Cloud Computing}'' o ``Computación en la nube'' \cite{nube1} estará muy presente en este proyecto.

\section{Descripción del proyecto}

El proyecto consistirá en el desarrollo de una aplicación cliente-servidor desplegada en la nube, adaptable a cualquier tipo de \textit{front-end} \cite{frontback} o interfaz.

Este proyecto tendrá especial énfasis en el servidor o \textit{back-end} \cite{frontback}, pero sin olvidar la parte de interfaz, ya que se busca que los usuarios se enganchen a este juego.

Para la parte de servidor, se desarrollará una API REST\cite{api1} \cite{api2}\cite{api3}, que nos permitirá la comunicación desde la interfaz hasta los datos alojados en la nube, añadiendo la capa de abstracción necesaria para que la aplicación sea adaptable, en un futuro, a diferentes tipos de \textit{front-end} o interfaces. 

Esta aplicación se desarrollará mediante la filosofía o práctica DevOps \cite{devops1} \cite{devops2} \cite{devops3} \cite{devops4}. En este proyecto será fundamental el desarrollo ágil de software y el monitoreo en todas las fases de su desarrollo, desde la integración hasta el despliegue e implementación.

Este proyecto se llamará \textbf{Project X}, y estará alojado en la plataforma GitHub en un repositorio del mismo nombre que el proyecto \cite{proyectogithub}. Esta plataforma facilita enormemente el desarrollo ágil y la filosofía DevOps nombrada anteriormente, haciendo que sea fácil el seguimiento de todas y cada una de las modificaciones realizadas en el proyecto y su continua integración y despliegue.

\section{Licencia}

Este proyecto y su memoria están liberados bajo la licencia GNU\footnote{https://es.wikipedia.org/wiki/GNU\_General\_Public\_License}.

\section{Estructura de la memoria}

\begin{itemize}
    \item \textbf{Capítulo 1. Introducción}: el presente capítulo, es una breve introducción al trabajo realizado. 
    \item \textbf{Capítulo 2. Gestión y planificación del proyecto}: cómo se ha realizado la gestión del proyecto, metodología, planificación, riesgos y costes.
    \item \textbf{Capítulo 3. Especificación de requisitos}: descripción de los requisitos e historias de usuario del proyecto.
    \item \textbf{Capítulo 4. Análisis}: análisis de los requerimientos del sistema, tecnología y herramientas de desarrollo.
    \item \textbf{Capítulo 5. Diseño}: diseño de la aplicación, arquitectura y funcionamiento general.
    \item \textbf{Capítulo 6. Implementación}: implementación de los aspectos descritos en el capítulo de diseño anterior.
    \item \textbf{Capítulo 7. Pruebas}: verificación del software y validación del proyecto mediante tests unitarios.
    \item \textbf{Capítulo 8. Conclusiones y posibles ampliaciones}: conclusiones finales del proyecto, posibles ampliaciones y propuesta de mejoras.
\end{itemize}