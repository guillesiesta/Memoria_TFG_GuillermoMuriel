\chapter{Introducción}
El ser humano es considerado un ser curioso, inquieto y siempre en busca de la aventura. No hay más que pararse a observar a un niño con un año de edad y comprobarlo.

Nuestra especie necesita retos a los que enfrentarse para crecer y mejorar. En definitiva, abrir la mente para evolucionar.

El mundo en el que vivimos nos plantea diariamente una serie de obstáculos que tenemos que superar día a día. Este trabajo, se realiza la mayor parte de las veces de una manera mecánica. Es decir, durante esa actividad, el cerebro no está usándose a pleno rendimiento. No estamos motivados.

Sin embargo, con la aparición de los juegos, al ser humano se le ofrece una actividad distinta, se le permite crujir su rutina.

Volviendo al origen de nuestra especie, nos adentramos en nuestra principal característica evolutiva, la resolución de problemas, tan fundamental para el desarrollo de nuestro intelecto.

He aquí la idea principal del proyecto, que no es otra que la de generar esa posibilidad que ayude al ser humano a su desarrollo. Todo esto apoyado mediante la creación de un juego conversacional. Gracias al cual se planteen y resuelvan acertijos de los más ingeniosos. 

Acertijos propuestos por individuos con alma de detective y resueltos por los mismos.

\section{Motivación y objetivo}

El poder colaborar en el desarrollo intelectual y en el entretenimiento de las personas es motivación más que suficiente para embarcarse en un proyecto de tal magnitud y dificultad.

Actualmente, cuando se decide crear un juego conversacional, se tienen al alcance de la mano, a través de internet, infinitas posibilidades para su creación.

El principal objetivo de este juego es que sea desarrollado con de una manera actual, teniendo muy en cuenta filosofías de desarrollo de software ágiles y con eje fundamental el software libre.

Este juego deberá ser accesible desde cualquier plataforma, por lo que para ello será fundamental su desarrollo mediante un servicio en la nube, es decir, el paradigma del 'Cloud Computing' o 'Computación en la nube' estará muy presente en este proyecto.

\section{Descripción del proyecto}

\cite{knuthwebsite}
Hablar de que va a ir el proyecto, cloud computing, enfoque de los juegos conversacionales. Enfoque DevOPs, desarrollo basado en pruebas, software libre, licencias.

\section{Estructura de la memoria}
