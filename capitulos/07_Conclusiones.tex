\chapter{Conclusiones y posibles ampliaciones}

El objetivo de este proyecto fue que la comunidad obtuviera un beneficio intelectual mediante la creación de una aplicación que pudiera crear una incertidumbre en el ser humano y que éste usara el intelecto para resolverla, para así crujir su rutina y desarrollar ese espíritu inquieto y aventurero que posee nuestra especie.

En resumen, este proyecto es totalmente apto para la \textbf{gestión de un juego conversacional}. Gracias la planificación que definió el esquema temporal en el que se desarrollaría el mismo y gracias a la correcta estimación finalización de cada uno de los \textit{sprints}, se puede confirmar que el tiempo fue el necesario para su correcto desarrollo.

Teniendo en cuenta los distintos requisitos recogidos en la sección tal podemos destacar los siguientes puntos:

El diseño modular de la aplicación permite que cada uno de los componentes presente un bajo acoplamiento, permitiendo la reutilización de cada uno de los módulos y, lo que es más interesante, la ampliación de la aplicación para añadir nuevas funcionalidades. 

Por otro lado, la metodología y la filosofía de trabajo elegidas fueron acertadas teniendo en cuenta la necesidad de validación continua, y los acotamientos temporales a los que fue sometido el proyecto.

Previo a un estricto proceso de investigación en el que se exploraron las distintas alternativas a aplicar en cada uno de los bloques y su posterior comparativa para así escoger aquella tecnología que mejor se adaptara a nuestro proyecto, nos lleva a que, junto con las tecnologías usadas en cada uno de los bloques o módulos desarrollados, se ha podido desplegar en la nube una aplicación, con una interfaz intuitiva y atractiva que enganche al usuario, (que hemos llamado Riddling, y que se puede acceder a ésta a través de la siguiente dirección: \textbf{https://riddling.herokuapp.com/}), de una manera satisfactoria.

Se completaron casi todas las historias de usuario, a excepción de las historias de usuario HU-07 y HU-12+1 que, como se notificó, al ser estas de una prioridad de nivel bajo, esta funcionalidad se añadiría única y exclusivamente en el caso de que sobrara tiempo en el \textit{sprint} correspondiente. Sin embargo, la no implementación de estos requisitos no suponen un inconveniente importante a la hora de la ejecución de la aplicación y su correcto funcionamiento. 

Finalmente, se de muestra que un proyecto como este se puede desarrollar con precio asequible y, sobre todo, gracias al uso del software libre, se puede crear una aplicación compleja que reúna distintos tipos de lenguajes, filosofías y tecnologías y, que además entretenga a sus usuarios.

\section{Posibles ampliaciones}

Este proyecto se encuentra en una fase inicial de desarrollo si lo comparamos con cualquier otra aplicación comercial desarrollada por una empresa con recursos ilimitados, sobre todo, recursos tales como dinero y tiempo.

Si se dispusiera de más tiempo y recursos, a la aplicación se le podrían añadir, entre otras tantas funciones, aquellas que la conviertan en una plataforma con perfil de red social orientada a acertijos, por ejemplo:

\begin{itemize}
    \item Posibilidad de que un usuario modifique sus datos.
    \item Creación de apartado de relaciones de amistad entre usuarios.
    \item Sistema de notificaciones al usuario creador de un acertijo en el caso de que se haya propuesto una solución al mismo.
    \item Mejora del diseño visual de la interfaz.
    \item Posibilidad de sincronización a los usuarios con su perfil de Facebook y demás redes sociales.
    \item Creación de temáticas distintas para agrupar los acertijos y así facilitar la búsqueda de los mismos por los usuarios.
    \item Notificar al usuario que propuso una respuesta a un acertijo de que ésta ha sido puntuada.
    \item Realización de ránkings en los que mostrar, por ejemplo 
        \begin{itemize}
            \item Acertijos más difíciles.
            \item Usuarios con porcentajes más altos de acertijos solucionados.
            \item Amigos con porcentajes más altos de acertijos solucionados.
            \item Acertijo del mes.
        \end{itemize}
\end{itemize}

Como se puede observar, las posibilidades de ampliación de este proyecto son inmensas. El límite de éste se encuentra en la imaginación.