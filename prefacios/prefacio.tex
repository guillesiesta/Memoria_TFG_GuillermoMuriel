\chapter*{}
%\thispagestyle{empty}
%\cleardoublepage

%\thispagestyle{empty}

\input{portada/portada_2}



\cleardoublepage
\thispagestyle{empty}

\begin{center}
{\large\bfseries \myTitle}\\
\end{center}
\begin{center}
\myName\\
\end{center}

%\vspace{0.7cm}
\noindent{\textbf{Palabras clave}: juegos conversacionales, aplicación, computación en la nube, API REST, SCRUM, DevOps, arquitectura cliente-servidor.}\\

\vspace{0.7cm}
\noindent{\textbf{Resumen}}\\

Los juegos conversacionales necesitan una gestión del servidor que sea capaz de gestionar eficientemente diferentes clientes, navegadores y clientes conversacionales. Además, es necesario presentarlos de forma atractiva para que los usuarios se enganchen en este tipo de juegos.

En este proyecto se programará una aplicación cliente-servidor, con especial énfasis en el servidor, pero con diferentes tipos de front-end; a la vez, el servidor se podrá usar como servicio web y será adaptable a diferentes tipos de juegos.
\cleardoublepage


\thispagestyle{empty}


\begin{center}
{\large\bfseries Management of conversational games: An application in the cloud}\\
\end{center}
\begin{center}
\myName\\
\end{center}

%\vspace{0.7cm}
\noindent{\textbf{Keywords}: conversational game, application, cloud computing, API REST, SCRUM, DevOps, client–server model.}\\

\vspace{0.7cm}
\noindent{\textbf{Abstract}}\\

The conversational games need a special management of the server. This has to  manage  efficiently different clients, browsers and conversational clients. Also is important to present them in an attractive way so that the users get hooked in this type of games.

In this project a client-server application will be programmed, with special emphasis on the server, but with different types of front-end; at the same time, the server can be used as a web service and will be adaptable to different types of games.
\chapter*{}
\thispagestyle{empty}

\noindent\rule[-1ex]{\textwidth}{2pt}\\[4.5ex]

Yo, \textbf{\myName}, alumno de la titulación Grado en Ingeniería Informática de la \textbf{Escuela Técnica Superior
de Ingenierías Informática y de Telecomunicación de la Universidad de Granada}, con DNI 47255733W, autorizo la
ubicación de la siguiente copia de mi Trabajo Fin de Grado en la biblioteca del centro para que pueda ser
consultada por las personas que lo deseen.

\vspace{6cm}

\noindent Fdo: \myName

\vspace{2cm}

\begin{flushright}
Granada a 7 de septiembre de 2018.
\end{flushright}


\chapter*{}
\thispagestyle{empty}

\noindent\rule[-1ex]{\textwidth}{2pt}\\[4.5ex]

D. \textbf{\myProf}, Profesor del Área de  Departamento de Arquitectura y Tecnología de Computadores del Departamento de Arquitectura y Tecnología de Computadores de la Universidad de Granada.

\vspace{0.5cm}


\vspace{0.5cm}

\textbf{Informa:}

\vspace{0.5cm}

Que el presente trabajo, titulado \textit{\textbf{\myTitle, Una aplicación en la nube}},
ha sido realizado bajo su supervisión por \textbf{\myName}, y autorizamos la defensa de dicho trabajo ante el tribunal
que corresponda.

\vspace{0.5cm}

Y para que conste, expiden y firman el presente informe en Granada a 7 de septiembre de 2018.

\vspace{1cm}

\textbf{El director:}

\vspace{5cm}

\noindent \textbf{\myProf \ \ \ \ \ }

\chapter*{Agradecimientos}
\thispagestyle{empty}

       \vspace{1cm}


Poner aquí agradecimientos...

