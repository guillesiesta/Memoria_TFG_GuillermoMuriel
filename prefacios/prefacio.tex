\chapter*{}
%\thispagestyle{empty}
%\cleardoublepage

%\thispagestyle{empty}

\input{portada/portada_2}



\cleardoublepage
\thispagestyle{empty}

\begin{center}
{\large\bfseries \myTitle}\\
\end{center}
\begin{center}
\myName\\
\end{center}

%\vspace{0.7cm}
\noindent{\textbf{Palabras clave}: juegos conversacionales, aplicación, computación en la nube, API REST, SCRUM, DevOps, arquitectura cliente-servidor.}\\

\vspace{0.7cm}
\noindent{\textbf{Resumen}}\\

Los juegos conversacionales necesitan una gestión del servidor que sea capaz de gestionar eficientemente diferentes clientes, navegadores y clientes conversacionales. Además, es necesario presentarlos de forma atractiva para que los usuarios se enganchen en este tipo de juegos.

En este proyecto se programará una aplicación cliente-servidor, con especial énfasis en el servidor, pero con diferentes tipos de front-end; a la vez, el servidor se podrá usar como servicio web y será adaptable a diferentes tipos de juegos.
\cleardoublepage


\thispagestyle{empty}


\begin{center}
{\large\bfseries Management of conversational games: An application in the cloud}\\
\end{center}
\begin{center}
\myName\\
\end{center}

%\vspace{0.7cm}
\noindent{\textbf{Keywords}: conversational game, application, cloud computing, API REST, SCRUM, DevOps, client–server model.}\\

\vspace{0.7cm}
\noindent{\textbf{Abstract}}\\

All conversational games need a good management of the server that stores them. This server must efficiently manage different accesses of different clients, browsers and conversational clients. In addition, the information must be presented in an attractive way with the aim that users adore playing these types of games.

In this project a client-server application will be programmed, with special emphasis on the server, but it will be able to addapt with different types of front-end; at the same time, the server can be used as a web service and will be adaptable to different types of games.
\chapter*{}
\thispagestyle{empty}

\noindent\rule[-1ex]{\textwidth}{2pt}\\[4.5ex]

Yo, \textbf{\myName}, alumno de la titulación Grado en Ingeniería Informática de la \textbf{Escuela Técnica Superior
de Ingenierías Informática y de Telecomunicación de la Universidad de Granada}, con DNI 47255733W, autorizo la
ubicación de la siguiente copia de mi Trabajo Fin de Grado en la biblioteca del centro para que pueda ser
consultada por las personas que lo deseen.

\vspace{6cm}

\noindent Fdo: \myName

\vspace{2cm}

\begin{flushright}
Granada a 7 de septiembre de 2018.
\end{flushright}


\chapter*{}
\thispagestyle{empty}

\noindent\rule[-1ex]{\textwidth}{2pt}\\[4.5ex]

D. \textbf{\myProf}, Profesor del Área de  Departamento de Arquitectura y Tecnología de Computadores del Departamento de Arquitectura y Tecnología de Computadores de la Universidad de Granada.

\vspace{0.5cm}


\vspace{0.5cm}

\textbf{Informa:}

\vspace{0.5cm}

Que el presente trabajo, titulado \textit{\textbf{\myTitle, Una aplicación en la nube}},
ha sido realizado bajo su supervisión por \textbf{\myName}, y autorizamos la defensa de dicho trabajo ante el tribunal
que corresponda.

\vspace{0.5cm}

Y para que conste, expiden y firman el presente informe en Granada a 7 de septiembre de 2018.

\vspace{1cm}

\textbf{El director:}

\vspace{5cm}

\noindent \textbf{\myProf \ \ \ \ \ }

\chapter*{Agradecimientos}
\thispagestyle{empty}

       \vspace{1cm}
Primeramente, me gustaría dar las gracias a la Universidad de Granada y más en concreto a la ETSIIT. Muchas gracias por exprimirme al máximo para sacar lo mejor de mi. Gracias por sentar las bases de mi carrera.

Gracias también, de verdad, a todas y cada una de las personas que han dicho que no sería capaz de acabar mi carrera y, así sembrar la incertidumbre y la desconfianza necesarias para crear un entorno hostil a mi alrededor. Quiero que sepáis que una parte de mi motivación ha sido el poder en un futuro confirmar vuestro estrepitoso ridículo al subestimarme. 

Además, me gustaría dar las gracias aquellos que, todos estos años, han sembrado tanta energía positiva a mi alrededor que necesitaré décadas para devolverla:

\begin{itemize}
    \item A Vicky, porque al principio, cuando todo se puso oscuro, supo encontrar la luz.
    \item A David Cubero, por esa infancia juntos y mil cosas más.
    \item A Peska y Dolo, los mejores compañeros de piso y amigos habidos y por haber. Gracias por hacerme sentir así siempre.
    \item A mis amigos Iván, Villegas, Veleta y Pastoro. Por estar siempre ahí y haberme aguantado tantos años, os estaré eternamente agradecido.
    \item A Manzano, por enseñarme la diferencia entre tener confianza y estar confiado.
    \item A Iván Sevillano y Diego, por ese año de Erasmus, fue sin duda nuestro año.
    \item A  Curro, Pepe, Raúl, Juanjo, Chenchu, Javi y Montero. Por ser tan buenos conmigo y hacerme sentir como en casa siempre que volvía tras estar tiempo fuera. 
\end{itemize}

Por último, y más importante. Dar las gracias a mi familia por haber confiado siempre en mi. Papá, Mamá, Paula, todo lo que soy, es gracias a vosotros.

A Mary Luz, la persona que ha conseguido que me vea capaz de comerme el mundo y de demostrarme que lo imposible es posible. Gracias por hacerme recordar quien era y por crear en mi una ambición por vivir solamente aplicable a los personajes de historias de fantasía. Te estaré eternamente agradecido.